\subsection{Keywords and Reserved Words}

\subsubsection{Keywords}
\begin{center}
  \begin{tabulary}{\linewidth}{|C|C|}
    \hline
    \bd{Keyword} & \bd{Definition} \\
    \hline
    \hline
    \cd{<datatype>\_Style\qquad\qquad\qquad\qquad\qquad} & Used for defining custom data types for the variable, allowing to create structured data. (e.g. \cd{int\_Style}, \cd{str\_Style}) \\
    \hline
    \cd{AllToWhile} & Versatile keyword combining the functionality of both \cd{for} and \cd{while} loops, enabling looping through code blocks in many ways. \\
    \hline
    \cd{SpeakNow} & Used for displaying output to the console.\\
    \hline
    \cd{Breathe} & Used to exit or terminate a loop, allowing to escape the current iteration and continue with the next part of your program. \\
    \hline
    \cd{Mine} & Use to take input from the user. \\
    \hline
    \cd{When} & Used for conditional statement, allowing to execute a block of code if a condition is true.\\
    \hline
    \cd{Thats} & Used in conjunction with a \cd{When} statement and specifies an alternative block of code to be executed if the \cd{When} condition is \cd{false}.\\
    \hline
    \cd{ThatsWhen} & Used to specify an additional condition to check if the previous \cd{When} or \cd{Thats} conditions are false. If the \cd{When} or preceding \cd{Thats} conditions are false, the \cd{ThatsWhen} condition is checked, and if it's true, the associated block of code is executed. \\
    \hline
    & \cd{Getaway-Car} is used for exception handling. It allows you to enclose a block of code that might raise exceptions and provides a way to handle those exceptions without causing the entire program to terminate abruptly. \\
    \cd{Getaway} & \cd{Getaway} block is a block of code where you suspect an exception might occur. \\
    \cd{Car} & \cd{Car} is a block that follows the \cd{Getaway} block and used to catch and handle specific types of exceptions that might be thrown in the \cd{Getaway} block. \\
    \hline
    \cd{BackToDecember} & Used to exit a method and optionally return a value to the caller. It is commonly used in methods to provide a result back to the calling code. \\
    \hline
    & \cd{Dear-John} is used for multi-branching based on the value of an expression. Unlike switch statements from other languages, \cd{Dear-John} doesn’t support fall through cases. \\
    \cd{Dear} & \cd{Dear} statement evaluates the expression and, depending on its value, executes the code block associated with the matching case. \\
    \cd{John} & \cd{John} keyword is used to define individual cases. \\
    \cd{Closure} & \cd{Closure} is used to specify a default case within a \cd{Dear-John} statement, providing a block of code to be executed when none of the preceding conditions on John are met. \\
    \hline
  \end{tabulary}
\end{center}

\subsubsection{Reserved Words}
\begin{center}
  \begin{tabulary}{\linewidth}{|C|C|}
    \hline
    \bd{Reserved Word} & \bd{Definition} \\
    \hline
    \hline
    \cd{BlankSpace\qquad\qquad} & Denotes a \cd{null} or ``empty'' value, typically used for uninitialized variable or when data is absent. \\
    \hline
    \cd{Clean} & Clears the console screen. \\
    \hline
    \cd{Evermore} & Used to skip the rest of the current iteration of a loop and proceed to the next iteration, based on a specified condition. \\
    \hline
    \cd{The1} & Used to represent boolean value \cd{True}. \\
    \hline
    \cd{The0} & Used to represent boolean value \cd{False}.\\
    \hline
  \end{tabulary}
\end{center}
